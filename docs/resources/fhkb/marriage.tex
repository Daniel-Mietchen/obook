\chapter{Exercise \arabic{excounter}}
\addtocounter{excounter}{1}

\steps{In-laws: Modelling partnerships}{
\item Add the class \con{Partnership} to the \fhkb as a sibling of \person and make it disjoint with its primitive siblings.
\item Create the object properties \con{hasParticipant}, \con{hasMaleParticipant} and \con{hasFemaleParticipant} in the obvious object hierarchy, along with their inverses. \con{Partnership} is their common domain and add the obvious ranges to these properties.
\item Add a restriction of \con{hasParticipant min 2 Person} to the \con{Marriage} class.
\item Create the object properties \con{hasSpouse}, \con{hasWife} and \con{hasHusband} and inverses where appropriate (or use property characteristics when they are not). Use sub-property chains to infer when two individuals are husband and wife. \ds and \mgs were married in 1958; create an individual for this marriage (you can add a \con{hasMarriageYear} data property if you wish). John Bright and Joyce Gosport were married in 1954. Add another individual for this marriage.
\item Run the reasoner and ask DL queries to test what you have done.
\item Create new object properties for in-laws-brother-in-law, sister-in-law (hint: these last two have possible sub-property chains)  and sibling-in-law. You can also now add properties to find uncles- and aunts-in-law.
\item Run the reasoner, and ask DL queries to confirm that it all works.
\item Add these  two property hierarchies to the main object property hierarchy, reason and look at it.
}