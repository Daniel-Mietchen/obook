\chapter{Exercise \arabic{excounter}}
\addtocounter{excounter}{1}

\steps{Domains and ranges}{
\item Run the reasoner and ask DL queries about who is a \man,  \woman and \person; note the answers.
\item  Add the domain \person and range \man to \con{hasFather} and add the domain \person and range \woman to \con{hasMother}.
\item Run the reasoner and inspect the object hierarchy to see what the reasoner has done with the domains and ranges of the other properties; make any changes that are necessary.
\item Ask DL queries about who is a \man,  \woman and \person; note for whom we don't have a specific sex.
\item Write the following defined classes (using the \con{equivalentTo:} construct):
\begin{enumerate}
\item  Parent---All the individuals that are parents.
\item A \con{Grandparent} class that uses the \con{isGrandparentOf} property and \con{Grandparent2} that uses \con{EquivalentTo: Person and (isParentOf some (Person and isParentOf some Person))}. 
\end{enumerate}
Run the reasoner, look to see where the classes are placed in the
hierarchy and work out why.  

\item Add the restrictions \con{hasFather some Man} and \con{hasMother some Woman} to \person. Run the reasoner and ask which individuals have a mother. Also ask the DL query for which individuals have a grandmother. Ask for explanations.
\item Test the \con{hasBloodRelation} property again. This time you should see it working. this is because we've now said everyone has a mother and everyone has a father, therefore everyone has a parent and thus an ancester. If everyone can have an ancestor,  then everyone is someone's descendent. 

}