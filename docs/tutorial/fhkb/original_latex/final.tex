\chapter{Final remarks}
\label{chap:final}

If you have done all the tasks within this tutorial, then you will have touched most parts of \owlii. Unusually for most uses of OWL we have concentrated on individuals, rather than just on the TBox. One note of warning -- the full \fhkb has some 450 members of the Bright family and takes a reasonably long time to classify, even on a sensible machine. The \fhkb is not scalable in its current form.\herebedragons

One reason for this is that we have deliberately maximised inference. We have attempted not to explicitly type the individuals, but drive that through domain and range constraints. We are making the property hierarchy do lots of work. For the individual \rds, we only have a couple of assertions, but we infer some 1\,500 facts between \rds and other named individuals in the \fhkb -- displaying this in \protege causes problems. We have various complex classes in the TBox and so on.

We probably do not wish to drive a genealogical application using an \fhkb in this form. Its purpose is educational. It touches most of \owlii and shows a lot of what it can do, but also a considerable amount of what it cannot do. As inference is maximised, the \fhkb breaks most of the \owlii reasoners at the time of writing.\herebedragons However, it serves its role to teach about \owlii.

\owlii on its own and using it in this style, really does not work for family history. We have seen that siblings and cousins cause problems. rules in varius  forms can do this kind of thing easily---it is one of  the primary examples for learning about Prolog. Nevertheless, the \fhkb does show how much inference between named individuals can be driven from a few fact assertions and a property hierarchy. Assuming a powerful enough reasoner and the ability to deal with many individuals, it would be possible  to make a family history application using the \fhkb; as long as one hid the long and sometimes complex   queries and manipulations that would be necessary to `prune' some of the `extra' facts found about individuals. However, the \fhkb does usefully show the power of \owlii, touch a great deal of the language and demonstrate some of its limitations.


%\todo{haven't done simple and complex properties}
