\chapter{Adding some Individuals to the \fhkb}
\label{chap:indiv}

In this chapter we will start by creating a fresh OWL ontology and adding some individuals that will be surrogates for people in the \fhkb. In particular you will:

\begin{enumerate}
\item Create a new OWL ontology for the \fhkb;
\item Add some individuals that will stand for members of the \stevens family.
%of the class \con{Person};
%\item Assert them to be members of the class \con{Person};
\item Describe parentage of people.
\item Add some facts to specific individuals as to their parentage;
\item See the reasoner doing some work.
\item At the moment we will ignore sex; sex will not happen until Chapter~\ref{chap:person}.
\end{enumerate}

\section{A World of Objects}

The `world'\footnote{we use `world' as a synonym of `field of interest' or `domain'. `World' does not restrict us to modelling the physical world outside our consciousness. } or field of interest we model in an ontology is made up of objects or individuals. Such objects include, but are not limited to:
\begin{itemize}
\item People, their pets, the pizzas they eat;
\item The processes of cooking pizzas, living, running, jumping, undertaking a journey;
\item The spaces within a room, a bowl, an artery;
\item The attributes of things such as colour, dimensions, speed, shape of various objects;
\item Boundaries, love, ideas, plans, hypotheses.
\end{itemize}
\noindent We observe these objects, either outside lying around in the world or in our heads. OWL is all about modelling such individuals. Whenever we make a statement in OWL, when we write down an axiom, we are making statements about individuals. When thinking about the axioms in an ontology it is best to think about the individuals involved, even if OWL individuals do not actually appear in the ontology. All through this tutorial we will always be returning to the individuals being described in order to help us understand what we are doing and to help us make decisions about how to do it.

\section{Asserting Parentage Facts}

%So far, through the description of the \person class, we know that all instances of the class \person have a mother and father; we  have seen this through the use of a DL query. We do not, however, know about the actual parents of people in the \fhkb.

%So far, we have done assertion of types on individuals; now we come to fact assertions. 

Biologically, everyone has parents; a mother and a father\footnote{Don't quibble; it's true enough here.}. The starting point for family history is parentage; we need to relate the family member objects by object properties. An object property relates two objects, in this case a child object with his or her mother or father object. To do this we need to create three object properties:

\steps{Creating object properties for parentage}{
\item Create a new ontology;
\item Create an object property \con{hasMother}; 
\item Create a property \con{isMotherOf} and give \con{hasMother} the \texttt{InverseOf:} \con{isMotherOf};
\item Do the same for the property \con{hasFather};
\item Create a property \con{hasParent}; give it the obvious inverse;
\item Make \con{hasMother} and \con{hasFather} sub-properties of \con{hasParent}.
\item Run the reasoner and look at the property hierarchy.
}

Note how the reasoner has automatically completed the sub-hierarchy for \con{isParentOf}: \con{isMotherOf} and \con{isFatherOf} are inferred to be sub-properties of \con{isParentOf}. 

The OWL snippet below shows some parentage fact assertions on an individual. Note that rather than being assertions to an anonymous individual via some class, we are giving an assertion to a named individual. 
%Of course, we have asserted parentage `anonymously' at  the class level -- as a \emph{universal} statement about each and every instance of the class \person having a mother and a father.
\\\\
\owlcode{
Individual: grant\_plinth

	Facts:
		hasFather mr\_plinth,
		hasMother mrs\_plinth
}

\steps{Create the ABox}{
%\item Go back to the \con{fhkb.owl} (the state of the \fhkb at the end of the last chapter). 
\item Using the information in Table~\ref{tab:familydata} (see appendix) about parentage (so the columns about fathers and mothers), enter the fact assertions for the people which appear in rows shaded in grey. We will only use the \con{hasMother} and \con{hasFather} properties in our fact assertions. You do not need to assert names and birth years yet. This exercise will require you to create an individual for every person we want to talk about, using the Firstname\_Secondname\_Familyname\_Birthyear pattern, as for example in \irds. 
}


\snapshot{While asserting facts about all individuals in the \fhkb will be a bit tedious at times, it might be useful to at least do the task for a subset of the family members. For the impatient reader, there is a convenience snapshot of the ontology including the raw individuals available at \fhkbhome.}

\note{If you are working with \protege, you may want to look at the Matrix plugin for \protege at this point. The plugin allows you to add individuals quickly in the form of a regular table, and can significantly reduce the effort of adding any type of entity to the ontology. In order to install the matrix plugin, open \protege and go to File >> Check for plugins. Select the `Matrix Views' plugin. Click install, wait until the the installation is confirmed, close and re-open \protege; go to the `Window' menu item, select `Tabs' and add the `Individuals matrix'.}

Now do the following:
\steps{DL queries}{
\item Classify the \fhkb.
\item Issue the DL query \con{hasFather value \ids} and look at the answers (remember to check the respective checkbox in \protege to include individuals in your query results).
\item Issue the DL query \con{isFatherOf value \irds}. Look at the answers.
\item Look at the entailed facts on \irds.
}

You should find the following:
\begin{itemize}
\item David Bright (1934) is the father of Robert David Bright (1965) and Richard John Bright (1962).
\item Robert David Bright (1965)  has David Bright 1934 as a parent.
\end{itemize}

Since we have said that \con{isFatherOf} has an inverse of \con{hasFather}, and we have asserted that \irds \con{hasFather} \ids{}, we have a simple entailment that \ids \con{isFatherOf} \irds{}. So, without asserting the \con{isFatherOf} facts, we have been able to ask and get answers for that DL query.

As we asserted that \irds \con{hasFather} \ids{}, we also infer that he \con{hasParent} \ids{}; this is because \con{hasParent} is the super-property of \con{hasFather} and the sub-property implies the super-property. This works all the way up the property tree until \con{topObjectProperty}, so all individuals are related by \con{topObjectProperty}---this is always true. This implication `upwards' is the way to interpret how the property hierarchies work. 

%\section{Should I say it Just Because It's True?}

\section{Summary}
We have now covered the basics of dealing with individuals in OWL ontologies. We have set up some properties, but without domains, ranges, appropriate characteristics and then arranged them in a hierarchy. From only a few assertions in our \fhkb, we can already infer many facts about an individual: Simple exploitation of inverses of properties and super-properties of the asserted properties.

We have also encountered some important principles:
\begin{itemize}
\item We get inverses for free.
\item The sub-property implies the super-property. So, \con{hasFather} implies the \con{hasParent} fact between individuals. This entailment of the super-property is very important and will drive much of the  inference we do with the \fhkb.
\item Upon reasoning we get the inverses of properties between named individuals for free.
\item Lots is still open. For example, we do not know the sex of individuals and what other children, other than those described, people in the \fhkb may have.
\end{itemize}

\expressivity{ALHI}

\ctime{26}{144}{7}